\documentclass[12pt, letterpaper]{article}
\usepackage[margin=3cm]{geometry}
\usepackage{multicol}
\usepackage{amsmath}
\usepackage{tikz}
\usepackage{pgfplots}
\usepackage{float}
\pgfplotsset{width=7.5cm}

\begin{document}

\begin{center}
	{\LARGE{Relatório 1º projeto ASA 2021/2022}}\\[\baselineskip]
\end{center}

\begin{flushleft}
	\textbf{Grupo:} t20\\
	\textbf{Alunos:} Guilherme Pascoal (99079), Pedro Lobo (99115)
\end{flushleft}


\section{Descrição do Problema e da Solução}

O 1º problema apresentado tem como objetivo determinar o número de subsequências
estritamente crescentes de tamanho máximo de uma sequência de inteiros e indicar
qual esse tamanho máximo.

Este problema pode ser resolvido recursivamente, apresentando subestrutura ótima.
$L$ corresponde à função que, para um dado índice, devolve o tamanho da maior
subsequência estritamente crescente até à posição $i$. $O$, de forma semelhante,
representa o número de ocorrências de subsequências de tamanho máximo até à
posição $i$.

\begin{center}
	\[
    L[i] = \left\{\begin{array}{lr}
		1, & \text{se } i = 1\\
		{1 + max(L[j] \mid 1 \leq j < i \land x_j \leq x_i)}, & \text{se } L[j] + 1 = L[i] \mid 1 \leq j < i \land x_i \geq x_j)\\
	\end{array}\right\}
	\]\\[\baselineskip]

	\[
    O[i] = \left\{\begin{array}{lr}
		O[j], & \text{se } L[j] + 1 > L[i]\\
		O[i] + O[j], & \text{se } L[j] + 1 = L[i]\\
        \end{array} \mid 1 \leq j < i \land x_i \geq x_j)\right\}
	\]\\[\baselineskip]

\end{center}

São utilizados dois vetores, para manter a informação descrita acima. Estes são
preenchidos sequencialmente. O maior tamanho será o máximo do vetor $L$ e o
número de ocorrências é dado pela soma dos valores de $O$ onde $L$ é máximo.\\[\baselineskip]


O 2º problema apresentado tem como objeto determinar o tamanho do maior subsequência
estritamente crescentes comum a duas sequências de inteiros.

Este problema pode ser resolvido recursivamente, apresentando subestrutura ótima.
O tamanho da maior subsequência pode ser definido como

\begin{center}
	\[
	L[i][j] = \left\{\begin{array}{lr}
		0, & \text{se } i = 0 \text{ ou } j = 0\\
		L[i-1][j], & \text{se } i,j > 0 \text{ e } x_1[i] \neq x_2[j]\\
		1 +	max(L[i-1][k] \mid 1 \leq k < j \land x_1[i] > x_2[j]), & \text{se } i,j > 0 \text{ e } x_1[i] = x_2[j]\\
        \end{array}\right\}
	\]\\[\baselineskip]
\end{center}

A matriz é preenchida linha a linha, coluna a coluna, sendo o resultado dado
pelo valor máximo da última linha preenchida.
No entanto, este problema pode ser resolvido recorrendo apenas a uma matriz $2
\times j$, uma vez que apenas são necessários valores da linha atual e da linha
anterior. Assim, o resultado do problema corresponde ao valor máximo da última
linha preenchida da matriz.


\pagebreak
\section{Análise Teórica}

\subsection{Problema 1}
\begin{itemize}
	\item Leitura dos dados de entrada. $\mathcal{O}(N)$
	\item Aplicação do algoritmo. $\mathcal{O}(N^2)$
	\item Apresentação do resultado. $\mathcal{O}(1)$
\end{itemize}
Complexidade global: $\mathcal{O}(N^2)$

\subsection{Problema 2}
\begin{itemize}
	\item Leitura do primeira sequência. $\mathcal{O}(N)$
	\item Leitura do segunda sequência. $\mathcal{O}(M)$
	\item Comparação do tamanho das sequências. $\mathcal{O}(1)$
	\item Inicialização do vetor auxiliar. $\mathcal{O}(N)$ ou $\mathcal{O}(M)$
	\item Aplicação do algoritmo. $\mathcal{O}(NM)$
	\item Procura do resultado. $\mathcal{O}(N)$ ou $\mathcal{O}(M)$
	\item Apresentação do resultado. $\mathcal{O}(1)$
\end{itemize}
Complexidade global: $\mathcal{O}(NM)$


\section{Avaliação Experimental dos Resultados}

Foram geradas sequências aleatórias de tamanho 1 a 100001 em incrementos de 1000.
Abaixo estão os gráficos do problema 1 e 2, respetivamente.

\begin{figure}[H]
	\centering
	\begin{tikzpicture}[scale=0.9]
		\begin{axis}[
			xmin=0, xmax=100050,
			ymin=0, ymax=35,
			xlabel={Tamanho da sequência},
			ylabel={Tempo(s)},
			]
			\addplot[blue,mark=square] table{../benchmarks/1.txt};
		\end{axis}
	\end{tikzpicture}
	\qquad
	\begin{tikzpicture}[scale=0.9]
		\begin{axis}[name=Tamanho,
			xmin=0, xmax=100050,
			ymin=0, ymax=15,
			xlabel={Tamanho da sequência},
			ylabel={Tempo(s)},
			]
			\addplot[blue, mark=square] table{../benchmarks/2.txt};
		\end{axis}
	\end{tikzpicture}
\end{figure}

Os gráficos estão de acordo com a análise teórica realizada.


\section{Bibliografia}

\begin{itemize}
	\item Thomas H. Cormen, Charles E. Leiserson, Ronald L. Rivest, Clifford Stein (2009)
	\emph{Introduction to Algorithms, Third Edition}, The MIT Press.
	\item \emph{https://wcipeg.com/wiki/Longest\_increasing\_subsequence}
\end{itemize}

\end{document}
