\documentclass[12pt, letterpaper]{article}
\usepackage[margin=3cm]{geometry}
\usepackage{amsmath}
\usepackage{tikz}
\usepackage{pgfplots}
\usepackage{float}
\pgfplotsset{width=7.5cm}

\begin{document}

\begin{center}
	{\LARGE{Relatório 1º projeto ASA 2021/2022}}\\[\baselineskip]
\end{center}

\begin{flushleft}
	\textbf{Grupo:} t20\\
	\textbf{Alunos:} Guilherme Pascoal (99079), Pedro Lobo (99115)
\end{flushleft}


\section{Problema 1}
\subsection{Descrição do Problema e da Solução}

O problema apresentado tem por objeto determinar o número de subsequências
estritamente crescentes de tamanho máximo de uma sequência de inteiros, bem
como indicar qual é esse tamanho máximo.

O problema pode ser resolvido recursivamente e apresenta sub-estrutura ótima.
O tamanho da maior subsequência, LIS, e o número de occorrências desta podem ser
definidos como

\begin{center}
	${LIS[i] = \{1 + max(LIS[j] \mid 1 \leq j < i \land x_j \leq x_i)}\}$\\
	\[
    OCC[i] = \left\{\begin{array}{lr}
		OCC[j], & \text{se } LIS[j] + 1 > LIS[i]\\
		OCC[i] + OCC[j], & \text{se } LIS[j] + 1 = LIS[i]\\
        \end{array} \mid 1 \leq j < i \land x_j \leq x_i)\right\}
	\]\\[\baselineskip]

\end{center}

São mantidos dois vetores, um para os tamanhos e outro para as sequências.
Estes são preenchidos sequencialmente. O maior tamanho será o máximo do vetor
LIS e o número de occorrências é a soma das ocorrências onde o tamanho
é máximo.


\subsection{Análise Teórica}

\begin{itemize}
	\item Leitura dos dados de entrada. $\mathcal{O}(N)$
\end{itemize}

\subsection{Avaliação Experimental dos Resultados}

\begin{figure}[H]
	\centering
	\begin{tikzpicture}
		\begin{axis}[
			xmin=0, xmax=100000,
			ymin=0, ymax=35,
			xlabel={Tamanho da sequência},
			ylabel={Tempo(s)},
			]
			\addplot[blue,mark=square] table{../benchmarks/1.txt};
		\end{axis}
	\end{tikzpicture}

\end{figure}

O gráfico está de acordo com a análise teórica prevista.


\section{Problema 2}
\subsection{Descrição do Problema e da Solução}

O problema apresentado tem por objeto determinar o tamanho do maior subsequência
estritamente crescentes comum a duas sequências de inteiros.

O problema pode ser resolvido recursivamente e apresenta sub-estrutura ótima.
O tamanho da maior subsequência, LIS, pode ser definido como

\begin{center}
\end{center}


\subsection{Análise Teórica}

\begin{itemize}
	\item Leitura dos dados de entrada. $\mathcal{O}(N)$
	\item Procura do mínimo da sequência. $\mathcal{O}(N)$
	\item Inicialização do vetor auxiliar. $\mathcal{O}(N)$
	\item Aplicação do algoritmo. $\mathcal{O}(N^2)$
	\item Apresentação do resultado. $\mathcal{O}(1)$
\end{itemize}
Complexidade global: $\mathcal{O}(N^2)$

\subsection{Avaliação Experimental dos Resultados}

\begin{figure}[H]
	\centering
	\begin{tikzpicture}
		\begin{axis}[name=Tamanho,
			xmin=0, xmax=100000,
			ymin=0, ymax=10,
			xlabel={Tamanho da sequência},
			ylabel={Tempo(s)},
			]
			\addplot[blue, mark=square] table{../benchmarks/2.txt};
		\end{axis}
	\end{tikzpicture}

\end{figure}

O gráfico está de acordo com a análise teórica prevista.

\end{document}
